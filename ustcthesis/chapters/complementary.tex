% !TeX root = ../main.tex

\chapter{Techinal caveats}
\section{Particle injection}

Injecting particles can be a tricky thing in numerical implementation. Currently we find three available approaches to do this and every approach provide some choices to tune the simulation. First and the simplest way is to not use injection at all: extend the simulation length, so that in a small simulation area we have continuous plasma flow through it. The second method is to initialize macro-particles in the cells behind the boundaries (outside the simulation domain). And in the PIC loop structure, the particle injection occurs after current projection on the grid, particle sorting and synchronizations. Injected macro-particles therefore do not contribute to the current and fields of the current iteration but they are taken into account in the diagnostics. This method is adopted in Smilei \citep{derouillatSmileiCollaborativeOpensource2018}. The third approach is the most intuitive: place a surface and inject the particles through the flux. This approach is implemented in WarpX \citep{vayWarpXNewExascale2017}. Although the first one seems to be the most computationally expensive one as it require to simulate a larger domain and keep track of all the particles, it is in fact the most fast one in modern highly paralleled computers. It only needs to generate the particles at the first time while the second and the third methods need to produce new particles at every timestep by sampling the particle distribution function. For the first method, we can use mesh refinement techniques to speed up the computation in the areas that we are not interested. If the second and third approaches is a must in the case of simulating the long time interaction or time-changing particle distributions, we recommend to inject both positively and negatively charged species at the same time to ensure a neutral plasma. And to strengthen neutrality, species may be created at the same positions. And if the particle momentum is drawn from a Maxwellian, using a random positionning instead of the regular one may be a better choice because regular positionning may induce numerical effects such as loss of charge and spurious field near the boundary. The reason is explained in the following figure \ref{fig:position}. The regular positionning works when injecting a drifting cold plasma with a drift velocity sufficiently high to let the particles entering the simulation domain.


\begin{figure}[ht]
    \centering
    {\includegraphics[width=0.8\textwidth]{particle_injector_regular_random}}
    \caption{Position choice (from Smilei Website)}
    \label{fig:position}  
\end{figure}


\section{Particle reflection at the boundary}

Even if we implement an open boundary condition, particle reflection can still occur at the boundary because of the discretization of time in numerical algorithms. As there is some randomness in positions and momentum, electrons and ions will slowly separate, thus creating random electric + magnetic field noise. When the plasma reaches the boundaries, it may happen that an electron is removed, but the ion is still in the box if the timestep is large enough to separate them. This will create a local artificial space-charge field that is later compensated by the ion leaving the box and short-lived electromagnetic noise. When particles come from far from boundary they get more separated over time, this noise becomes more and more important. As a consequence, we may reach a point where adding the noise from several particles is enough to reflect one other particle. Several ways exist to help mitigate this artificial effect like reducing temperature, using "regular spacing" to remove any randomness, staying far from boundaries to ignore the boundary effect, having many more particles or adopting a smaller timestep to reduce the noise. This most feasible way in our simulation is to use a smaller timestep by increasing the spatial resolution (number of cells). Because the Courant–Friedrichs–Lewy condition (CFL) constraint, the timestep is reduced consequently.