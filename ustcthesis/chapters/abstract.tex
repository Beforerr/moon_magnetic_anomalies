% !TeX root = ../main.tex

\ustcsetup{
  keywords  = {月球磁异常, 太阳风相互作用},
  keywords* = {lunar magnetic annomaly, solar wind interaction, particle-in-cell},
}

\begin{abstract}
  月球缺乏一个全球性的磁层和大气层, 当其围绕地球旋转时, 它直接暴露在周围的太阳风和/或磁层等离子体中. 此前, 月球被认为只是被动的吸收周围等离子体, 其下游因为这种吸收会形成一个几乎完全的称为月球尾迹的真空; 而月球近表面空间环境主要由月球表面与环境等离子体之间的相互作用决定. 然而, 最近的空间任务揭示了月球与太阳风之间更为复杂和多样的相互作用, 例如太阳风质子的反射和偏转以及围绕月球磁异常区域(lunar magnetic anomaly, LMA)的“微型磁层”和无碰撞激波. 这些新的观测激发了人们对月球等离子体环境研究的新兴趣. 在本论文中, 我们研究了太阳风和月球地壳磁异常之间的相互作用. 我们使用 particle-in-cell 方法来模拟研究嵌入在月球表土中的磁偶极子与太阳风之间相互作用这一简化但典型的场景. 我们证实了在一般的太阳风条件下和月球磁异常的条件下, 磁异常的强度确实足以阻止太阳风直接撞击月球表面, 并形成一个微型磁层结构. 对微型磁层和参数影响的分析, 有助于我们深入了解行星磁结构中的多尺度动力学物理过程.

\end{abstract}

\begin{abstract*}
  The Moon, charactered by an absence of magnetosphere and atmosphere, sits directly exposed to the surrounding solar wind and/or magnetospheric plasma as the Moon revolved around the Earth. Therefore, the lunar space environment is mainly shaped by the interaction between the lunar surface and the impinging plasma. Previously, the Moon was thought as a passive absorber of the solar wind, forming a nearly complete void downstream called lunar wake. However, recent mission revealed a more complicated and diverse interaction between the Moon and the solar wind, such as the reflection and deflection of solar wind protons, formation of "mini-magnetospheres" and collisionless shocks around lunar magnetic anomalies (LMAs). These new observation sparks a new interest to the lunar plasma environment research.
  
  In the present thesis, we investigate the interaction between the solar wind and lunar crustal magnetic anomalies. We use particle-in-cell simulation to study a simplified but typical scenario of the interaction between a magnetic dipolar embedded in the lunar regolith and solar wind. We confirm that LMAs may indeed be strong enough to stand off the solar wind from directly striking the lunar surface under typical solar wind conditions and form a mini-magnetosphere structure. The analysis of the miniature magnetosphere and parameter influence offers insight into kinetic multi-scale physical processes in the planetary magnetic structures.
\end{abstract*}
