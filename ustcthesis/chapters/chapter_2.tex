% !TeX root = ../main.tex

\chapter{Crustal magnetic anomaly interactions}

Given the small length scale and magnetic field strength of lunar crustal magnetic anomalies, one would expect their interaction with solar wind with more kinetic features. In this section, we present an overview of the observations related with LMAs, recent advances in simulation works and motivate our simulation through theory analysis.

\section{Observations}
    
Observation from various spacecraft have revealed a wealth of electromagnetic phenomena, including ion deflection and reflection of the incident solar wind, whistler and electrostatic waves, limb shocks, and steady electrostatic potentials above lunar magnetic anomalies.

\cite{wieserFirstObservationMinimagnetosphere2010} presented the first ENA image of a lunar magnetic anomaly, showing a clear enhancement around the crustal field and a decrease in the center. The decrease was ~20\%, thus not a complete void, but there could be voids present that are smaller than the ENA image resolution. Wieser et al. also observed a reduction in the ENA energy where the ENA flux was reduced. \cite{vorburgerEnergeticNeutralAtom2013} applied ENA imaging to the majority of the lunar surface, clearly observing reduction and deceleration of the solar wind at magnetic anomalies. The deceleration was also observed by orbital plasma instruments \citep{saitoSimultaneousObservationElectron2012}. Futaana et al. (2013) implemented a technique for measuring the surface potential using the observed ENA energy. These studies suggest surface potentials of ~200 V, which help to deflect the protons of the solar wind from the surface.

The solar wind protons are not only deflected into the nearby regions, but some are also deflected/reflected away from the Moon \citep{saitoInflightPerformanceInitial2010}. The reflected proton streams were observed to have temperatures of 100s eV, compared to the ~10 eV of the solar wind. Reflection rates were 10\% on average and were 50\% or higher at the strongest magnetic anomalies. 

Besides these observations, data from Kaguya and Chandrayaan suggests that in some cases \citep{wieserFirstObservationMinimagnetosphere2010}, \cite{kurataMinimagnetosphereReinerGamma2005}, lunar crustal magnetic fields may be strong enough to stand off the solar wind and generate a mini-magnetosphere shielding the surface. These mini-magnetosphere are typically identified with a density cavity structure and may be the smallest magnetosphere in the solar system.

\section{Simulation}

Thanks to the advance of computation power, the simulation efforts has evolved from magnetohydrodynamic to hybrid fluid / kinetic and now towards full kinetic plasma solvers. Earlier simulation have demonstrated that the interactions between lunar crustal magnetic anomalies and solar wind are highly non-adiabatic and multi-scale. The finite gyro-radius effect and charge-separation effect play a vital role in shaping the near surface plasma environment \citep{decaThreedimensionalFullkineticSimulation2016, decaGeneralMechanismDynamics2015, decaElectromagneticParticleinCellSimulations2014}.

\section{Analysis}


The solar wind itself exhibits a fascinating structure on a range of spatial and temporal scale as it escapes the hot solar atmosphere and expands out into the space \citep{owensSolarWindStructure2020}. The focus in this paper focus on the influence of the Moon, so a simplified model of undisturbed solar wind is assumed. Typical values of the solar wind parameters are as given to illustrate the key property of their interaction with solar wind: the interplanetary magnetic field (IMF) strength is 3.0 nT, the solar wind density is 5 cm−3, the solar wind speed is 400 km/s, the solar wind particles temperature is 10.0 eV \citep{haradaUpstreamWavesParticles2016}. Based on these parameters, we can calculate the following physical quantities important in plasma physics.

In the interplanetary space, the thermal speed of protons, is of the order of 50 km/s, much slower than the solar wind bulk speed. In contrast, the thermal speed of solar wind electrons is of the order of 2000 km/s and is much higher than the bulk speed. There are almost no solar wind protons moving in the direction of the sun while a large number of electrons moving toward the sun initially. Lunar surface's absorption and magnetic reflection modifies the distribution function of the impinging solar wind, producing a loss cone structure in the electrons' velocity space \citep{halekasSolarWindElectron2012}.

The gyroradius of solar proton in the undisturbed solar wind is of the order of 1000 km, larger than the typical length scale of lunar magnetic anomalies. On the other hand, the gyroradius of solar electrons is of the order of 1000 m. As a result, over the length scale of LMAs, the electrons in the flowing plasma are magnetized and protons (ions) are effectively unmagnetized. This bases our understanding to understand the structure of possible magnetosphere above LMA regions. As the solar wind approaches the lunar surface, the electrons are slowed, deflected and sometimes reflected by the magnetic structure. The ions on these scales, however, cannot respond as quickly as electrons to the changes in magnetic field, penetrating the magnetic barrier and bombard the lunar surfaces. This difference results in a charge separation, forming an electric field responsible for slowing and deflecting the solar wind ions. Kinetic effects become important in this interaction area.

Recent global maps of LMAs show a diverse magnetic configurations over low and high altitudes with a characteristics of elongated shapes and suggest that base-forming impacts influence the large-scale distribution of LMAs due to the impact demagnetization \citep{tsunakawaSurfaceVectorMapping2015}, \cite{mitchellGlobalMappingLunar2008}. Modelling of LMAs is a subject challenging on its own to study, here we adopt the dipole source approximation to model the LMAs. This is a reasonable assumption on central magnetic anomalies area in a small size and [Takahashi et al. (2014)]\citep{takahashiReorientationEarlyLunar2014} demonstrate that it will give similar result of magnetization directions of prism model. Using a simplifying assumption seems to lack the ability to reconstruct the realistic picture of the interaction. However, by isolating different physical mechanisms we could study the main processes and understand the influence of various parameters, which is impossible with observations in view of complex conditions they may encounter.


Assuming a dipolar source is located at $r_0$ underneath parallel to the lunar surface with the dipole moment $m$. We first use the magnetohydrodynamics (MHD) approximation to evaluate the size of a magnetic dipole in the solar wind. The indicator here is the the pressure equilibrium point $L$ as measured from the dipole center. In the equilibrium point, the dynamic pressure of the impinging solar wind is balanced by the local magnetic pressure of the plasma flow. In regions below the point, particle dynamics are significantly influenced by the local crustal magnetic field and the lunar surface, whereas in regions above the solar wind are disturbed by the reflected particles and waves that propagated upstream.

$$L=\left(\frac{\mu_{0} M_d^{2}}{16 \pi^{2} N_{0} m_{i} V_{\text {flow }}^{2}}\right)^{\frac{1}{6}}$$

For typical value $M_d = 2 \times 10^{13} A \cdot m^2, V_{\text {flow }} = 400 km / s , N_0 = 5 cm^{-3}$, this $L$ corresponds to a value of 37 km, equivalent to $0.35 ~ d_i$ (ion inertial length) or $0.25 ~ r_{g,proton}$ (ion gyro-radius under the quiet solar wind condition)


However, this doesn't necessarily lead to the formation of an obstacle boundary. Especially in the case of lunar crustal magnetic anomalies, the incoming protons have gyro-radii comparable to the scale of the magnetic obstacle, they may not manage to turn around before they have flown past the obstacle, or impacted the lunar surface above the source of the magnetic field. In such a case, no magnetopause forms. So simulation has to be performed to check the existence of the barrier structure.

Greenstadt study the formation condition of a magnetosphere by magnetic fields of scale sizes comparable to the proton gyroradius and submagnetospheric interactions (in which a magnetosphere does not fully form) when considering the solar wind interaction with magnetized asteroids. They present three conditions required for the formation of a mini-magnetosphere that successfully stands-off the solar wind: (1) The magnetic field must be strong enough from a pressure-balance point-of-view. (2) The magnetopause distance from the surface must be greater than the solar wind stopping distance, i.e., the vertical distance required to turn-around the solar wind plasma. (3) The lateral (horizontal) scale size of the magnetic obstacle must be large enough to exclude edge-effects. If these criteria are not met, the solar wind will fill-in the crustal field and be deflected but not creating a proper void. Nevertheless, such an interaction would create various electromagnetic noise. If the solar wind deceleration is large enough, a magnetosonic shock would form even without a void region. Whistler waves would also arise from the disturbance of the solar wind, possibly creating a standing whistler wave.

Back to the Moon, observations and simulations work suggest that a submagnetospheric type of interaction is most plausible, for most of the lunar magnetic anomalies, most of the time. However, in some cases, or perhaps commonly but on small scales, mini-magnetosphere may form. Lunar albedo swirls (bright features on the lunar surface at strong magnetic anomalies) may be indicative of small plasma voids, where surface weathering is reduced \citep{garrick-bethellSpectralPropertiesMagnetic2011}.
