% !TeX root = ../main.tex

\chapter{Summary and conclusion}

We have presented in detail two-dimensional realistic kinetic simulations of the solar wind interaction with lunar crustal magnetic anomalies (three-dimensional simulation with higher space and time resolution is still running on supercomputers when the authors write the paper, but a coarse 3D simulation has exhibited a similar structure to the 2D simulation). With the WarpX particle in cell code, we identied the mini-magnetosphere structure, and confirmed that LMAs may indeed be strong enough to stand off the solar wind from directly striking the lunar surface under typical solar wind conditions. Using a dipole model centered just below an absorbing surface representing the lunar surface under open boundary conditions, we described the interaction of the solar wind with an idealized LMA. The electron population is effectively magnetized throughout the whole simulatioin area, while the proton population is only magnetized when they becomes close to the magnetic dipole. And it is the charge separation due to the mass difference between the plasma species that sets up a electric field responsible for accelerating electrons and decelerating protons.

\section{Future work}

At various points in the simulation and analysis, one can make sidesteps and investigate further, e.g. to incorporate more physics processes like the effects to solar illumination and associated photoemission electrons, to set a pragmatic magnetic configuration and boundary topologies \citep{zimmermanKineticSimulationsKilometerscale2015}, to study the evolution of the mini-magnetosphere throughout the days and study plasma instabilities and wave-particle interactions. 

Note finally, that understanding LMAs and mini-magnetospheres is not only important for Lunar science. Mars are also found to have only crustal magnetization without a global magnetic field. Similarly, Ganymede’s magnetosphere formed inside the Jovian magnetosphere is considered a type of mini-magnetosphere. Simulating their interactions with the solar wind is within reach of the our simulation work.